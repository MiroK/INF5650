\documentclass[a4paper,11pt,titlepage]{article}
\pdfoptionpdfminorversion=5
\usepackage[utf8]{inputenc}
\usepackage{amsmath}
\usepackage{amsfonts}
%\usepackage{datetime}
%\usepackage{cancel}

\usepackage[square]{natbib}

\usepackage[table]{xcolor}
\usepackage[utf8]{inputenc}
\usepackage{graphicx}
\usepackage{graphics}
\usepackage{color}
\usepackage{ifpdf}
\ifpdf
\DeclareGraphicsRule{*}{mps}{*}{}
\fi

%\usepackage{blindtext}
%\usepackage[firstpage]{draftwatermark}
%\SetWatermarkText{Anna's personal copy!}
%\SetWatermarkScale{3}


\newcommand{\jump}[1]{\ensuremath{[\![#1]\!]} }
\newcommand{\VEC}[1]{\ensuremath{\textbf{#1}}}

\hoffset = 0pt
\voffset = 0pt
\oddsidemargin = 0pt
\headheight = 15pt
\footskip = 30pt
\topmargin = 0pt
\marginparsep = 5pt
\headsep =25pt
\marginparwidth = 54pt
\marginparpush = 7pt
\textheight = 621pt %621
\textwidth = 500pt

\title{Problem Convection-Diffusion 4}
\author{Miroslav Kuchta}
\date{2014}

\begin{document}
 


\textit{
Consider the problem 
\[
 \left\{ \begin{aligned}
	  -&\mu\Delta u + \VEC{v}\cdot\nabla u = f &\text{in}\,\,\Omega,\\
	  &u = g &\text{on}\,\,\Gamma_D = \{x\in\Omega\,|\,\VEC{v}\cdot\VEC{n} < 0\},\\
	  &\nabla u\cdot\VEC{n} = 0 &\text{on}\,\,\Gamma_N = \{x\in\Omega\,|\,\VEC{v}\cdot\VEC{n} \geq 0\}.\\
         \end{aligned}
 \right.
\]
Investigate whether the resulting bilinear form is coercive.}
\\

The following definitions will prove useful in the course of the solution:
\vskip 5pt
Let $H$ be a normed linear space with norm $||\cdot||_H$, $V\subset H$. Then 
\begin{itemize}
 \item[] A bilinear form $a(\cdot,\cdot)$ is coercive on $V$ if there exists $\alpha>0$ such that $a(v,v)\geq\alpha||v||^2_H$ for all $v\in H$.
 \item[] A bilinear form $s(\cdot,\cdot)$ is symmetric if $s(u,v)=s(v,u)$ for all $u,v\in H$.
\end{itemize}
\vskip 5pt

We now proceed to derive the bilinear form of the problem. Take $\phi\in C^{\infty}(\overline{\Omega})$ such that $\phi=0$ on $\Gamma_D$. Multiplying the governing equation with $\phi$
and integrating over the domain $\Omega$ yields an integral identity
\begin{equation}
\label{eq:x1}
\int_\Omega -\mu\nabla\cdot(\nabla u)\phi+\VEC{v}\cdot(\nabla u)\phi\,\mathrm{d}x = \int_\Omega f \phi\,\mathrm{d}x.
\end{equation}
Since $\mu>0$ is a constant the first term in (\ref{eq:x1}) can be manipulated by product rule and Gauss theorem as follows
\[
-\int_\Omega \mu\nabla\cdot(\nabla u)\phi \,\mathrm{d}x = -\int_\Omega \nabla\cdot(\mu\phi\nabla u)\,\mathrm{d}x + \int_\Omega \mu\nabla u\cdot \nabla\phi \,\mathrm{d}x
=-\int_{\partial\Omega=\Gamma_D\bigcup\Gamma_N} \mu\phi\nabla u\cdot\VEC{n}\,\mathrm{d}s + \int_\Omega \mu\nabla u\cdot \nabla\phi \,\mathrm{d}x.
\]
As part of the surface integral over $\Gamma_N$ vanishes due to property of the solution $\nabla u\cdot\VEC{n}=0$ on $\Gamma_N$ and the part over 
$\Gamma_D$ vanishes due to property of the test function $\phi=0$ on $\Gamma_D$ we get that $-\int_\Omega \mu\nabla\cdot(\nabla u)\phi \,\mathrm{d}x = 
\int_\Omega \mu\nabla u\cdot \nabla\phi \,\mathrm{d}x$. We now relax the regularity requirements on the test functions down to what is necessary in order
for the above integrals to be well defined and we state weak formulation of the problem:
\begin{equation*}
\text{Find } u\in H^1_{g,\Gamma_D}(\Omega)\text{ such that}
\int_\Omega \mu\nabla u\cdot \nabla v + \VEC{v}\cdot(\nabla u)v \,\mathrm{d}x= \int_\Omega f v \,\mathrm{d}x\,\,\,\forall v\in H^1_{0,\Gamma_D}(\Omega),
\end{equation*}
where $H^1_{g,\Gamma_D}(\Omega)=\left\{v\in H^1(\Omega)\,:\,v|_{\Gamma_D}=g\right\}$ and 
$H^1_{0,\Gamma_D}(\Omega)=\left\{v\in H^1(\Omega)\,:\,v|_{\Gamma_D}=0\right\}$. In this weak formulation $H^1_{g,\Gamma_D}(\Omega)$ is not a vector space
($u,v\in H^1_{g,\Gamma_D}(\Omega)$ does not imply $u+v\in H^1_{g,\Gamma_D}(\Omega)$). Also the formulation uses  different trial and test function spaces and thus
does not fit well into Lax-Milgram framework where the spaces are identical. To remedy for this we suppose that there is a function $\gamma_g\in H^1(\Omega)$ such 
that $\gamma_g|_{\Gamma_D}(\Omega)=g$. If we now define $u_0=u-\gamma_g$ we see that $u_0|_{\Gamma_D}=u|_{\Gamma_D}-\gamma_g|_{\Gamma_D}=g-g=0$ and thus $u_0\in H^1_{0,\Gamma_D}$.
If $u=u_0+\gamma_g$ is substituted into the integral in the above weak formulation an extra term $-\int_\Omega \mu\nabla \gamma_g\cdot \nabla v + \VEC{v}\cdot(\nabla \gamma_g)v$
appears on the right hand side. Taking this into account we state an equivalent weak formulation:
 \begin{equation*}
\text{Find } u_0\in V:=H^1_{0,\Gamma_D}(\Omega)\text{ such that }
a(u_0,v)=L(v)\,\,\,\forall v\in V,
\end{equation*}
where have defined the bilinear forms $s(\cdot,\cdot):V\times V\mapsto R$, $w(\cdot,\cdot):V\times V\mapsto R$, $a(\cdot,\cdot):V\times V\mapsto R$ and a linear form
$L(\cdot):V \mapsto R$ as $s(u,v)=\int_{\Omega} \mu\nabla u\cdot \nabla v\,\mathrm{d}x$, $c_{\VEC{v}}(u,v)=\int_{\Omega} \VEC{v}\cdot(\nabla u)v\,\mathrm{d}x$, 
$a(u,v)=c_{\VEC{v}}(u,v)+w(u,v)$ and $L(v)=\int_\Omega f v \,\mathrm{d}x - a(\gamma_g,v)$.

We now show that the bilinear form $a(\cdot,\cdot)$ is coercive on $V$. From the definition of coercivity we need to show that there exists $\alpha>0$ such that 
$a(u,u)\geq\alpha||u||^2_V$ for all $u\in V$. By Poincare inequality $||\nabla(\cdot)||_{L^2}$ is a norm on $V$. For brevity we donote this norm as $||\cdot||_{V}$. From the definition of $a(\cdot,\cdot)$ we have that $a(u,u)=s(u,u)+c_{\VEC{v}}(u,u)$ 
for all $u\in V$.

The bilinear form $s(\cdot,\cdot)$ is symmetric as follows readily from its definition but more importantly it is coercive on $V$ with the diffusion constant playing the role
of coercivity constant. To see this we calculate that
\begin{equation}
\label{eq:x2}
 s(u,u)=\int_{\Omega} \mu\nabla u\cdot \nabla v\,\mathrm{d}x =\mu \int_{\Omega} \nabla u\cdot \nabla u\,\mathrm{d}x=\mu ||u||^2_V
\end{equation}


The establish the properties of the bilinear form $c_{\VEC{v}}(\cdot,\cdot)$ we use the product rule and the Gauss theorem to calculate that
\begin{equation}
\label{eq:x3}
\begin{aligned}
 &c_{\VEC{v}}(u,u) = \int_\Omega \VEC{v}\cdot(\nabla u) u\,\mathrm{d}x=\frac{1}{2} \int_\Omega \VEC{v}\cdot(\nabla u^2) \,\mathrm{d}x = 
 \frac{1}{2} \int_{\Omega}\nabla\cdot(\VEC{v} u^2)\,\mathrm{d}x -  \frac{1}{2}\int_\Omega u^2 \nabla\cdot\VEC{v}\,\mathrm{d}x =\\
 & \frac{1}{2} \int_{\partial\Omega}\VEC{v}\cdot\VEC{n} u^2\,\mathrm{d}s -  \frac{1}{2}\int_\Omega u^2 \nabla\cdot\VEC{v}\,\mathrm{d}x.
\end{aligned}
 \end{equation}
If the flow is incompressible then $\nabla\cdot\VEC{v}=0$ and the corresponding volume integral in (\ref{eq:x3}) vanishes. The surface integral
can be divided into Neumann and Dirichlet part. The latter part vanishes since $u\in V$. Consequently we get from (\ref{eq:x3}) that  
$c_{\VEC{v}}(u,u)=\frac{1}{2} \int_{\Gamma_N}\VEC{v}\cdot\VEC{n} u^2\,\mathrm{d}s$. Since $u^2$ is positive and on the Neumann part
of the boundary we have from its definition as outflow boundary that $\VEC{v}\cdot\VEC{n}\geq0$ we see that $c_{\VEC{v}}(u,u)\geq0$ for all $u\in V$.

Putting the properties of both forms together we get that
\[
 a(u,u) = s(u,u) + c_{\VEC{v}}(u,u) \geq s(u,u) =\mu ||u||^2_V\,\,\,\forall u\in V.
\]
Thus we showed that the bilinear form $a(\cdot,\cdot)$ is coercive on $V$.

\end{document}
