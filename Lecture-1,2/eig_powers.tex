\documentclass[a4paper,11pt]{article}
\pdfoptionpdfminorversion=5
\usepackage[utf8]{inputenc}
\usepackage{amsmath}
\usepackage{amsfonts}
%\usepackage{datetime}
%\usepackage{cancel}

\usepackage[square]{natbib}

\usepackage[table]{xcolor}
\usepackage[utf8]{inputenc}
\usepackage{graphicx}
\usepackage{graphics}
\usepackage{color}
\usepackage{ifpdf}
\ifpdf
\DeclareGraphicsRule{*}{mps}{*}{}
\fi
\usepackage[utf8]{inputenc}

% Default fixed font does not support bold face
\DeclareFixedFont{\ttb}{T1}{txtt}{bx}{n}{12} % for bold
\DeclareFixedFont{\ttm}{T1}{txtt}{m}{n}{12}  % for normal

% Custom colors
\usepackage{color}
\definecolor{deepblue}{rgb}{0,0,0.5}
\definecolor{deepred}{rgb}{0.6,0,0}
\definecolor{deepgreen}{rgb}{0,0.5,0}

\usepackage{listings}

% Python style for highlighting
\newcommand\pythonstyle{\lstset{
language=Python,
basicstyle=\ttm,
otherkeywords={self},             % Add keywords here
keywordstyle=\ttb\color{deepblue},
emph={MyClass,__init__},          % Custom highlighting
emphstyle=\ttb\color{deepred},    % Custom highlighting style
stringstyle=\color{deepgreen},
frame=tb,                         % Any extra options here
showstringspaces=false            % 
}}


% Python environment
\lstnewenvironment{python}[1][]
{
\pythonstyle
\lstset{#1}
}
{}

% Python for external files
\newcommand\pythonexternal[2][]{{
\pythonstyle
\lstinputlisting[#1]{#2}}}

% Python for inline
\newcommand\pythoninline[1]{{\pythonstyle\lstinline!#1!}}


\newcommand{\jump}[1]{\ensuremath{[\![#1]\!]} }
\newcommand{\VEC}[1]{\ensuremath{\textbf{#1}}}

\hoffset = 0pt
\voffset = 0pt
\oddsidemargin = 0pt
\headheight = 15pt
\footskip = 30pt
\topmargin = 0pt
\marginparsep = 5pt
\headsep =25pt
\marginparwidth = 54pt
\marginparpush = 7pt
\textheight = 621pt %621
\textwidth = 500pt

\title{Computing Sobolev norms using powers of eigenvalues\\ of the
 stiffness matrix}
 \author{Miroslav Kuchta}
 \date{2014}

\begin{document}
\maketitle

Consider the eigenvalue problem: Find $u\in V$, $\lambda\in\mathbb{C}$ such that 
\begin{equation}
 \label{p1}
-u_{xx}=\lambda u
\end{equation}
where functions $u\in V$ are such that $u(0)=u(1)=0$. Using ansatz for the
solution, $u(x)=\exp(\alpha x)$, recover a characteristic polynomial
\[
-\alpha^2 = \lambda.
\]
Thus, the solution is a linear combination of $\sin(\sqrt{\lambda}x)$ and
$\cos(\sqrt{\lambda}x)$. The boundary condition $u(0)=0$ eliminates cosines,
while the other boundary condition requires eigenvalues
$\lambda=(k\pi)^2, k=1,2,...$. The corresponding eigenvectors are
$\phi_k(x)=\sin(k\pi x)$. The eigenvectors are perpendicular in a $L^2$-inner 
product $(\cdot,\cdot)$. Further, redefining $\phi_k=\sqrt{2}\phi_k$ makes the
vectors orthonormal. Note, that the vectors produced by octave/SLEPc are
normalized. Finally, the eigenvalues are real and positive. All the above
properties could have been established by observing that on $V$ the Laplacian
is symmetric and positive definite operator.

Since the eigenvalues are all distinct we get $n=\text{dim}(V_h)$ eigenvectors
that form orthonormal basis of $V_h$. Consequently every function $u$ can be 
represented in $V_h$ as $u(x)=c_k\phi_k(x)$, where $c_k$ are coefficients given
by $c_k=(u, \phi_k)$. 

Let us now compute the $L^2$-norm of $u$. Using the definition of the norm,
the representation of $u$ and orthonormality of the basis we get
\[
||u||^2 = (u, u) = (c_k \phi_k, c_i \phi_i)=c_k c_i \delta_{ki}=c_k^2=(u, \phi_k)^2.
\]
It remains to compute the inner product $(u, \phi_k)$. Denoting the nodal basis
function of $V_h$ as $\varphi_i, i=1, 2, ..., n$., functions $u$ and $\phi_k$
can be represented as
\[
\begin{aligned}[c]
&u(x)=U_i\varphi_i(x),\\
&\phi_k(x)=F^k_j\varphi_j(x).\\
\end{aligned}
\]
Thus the inner product of $u$ and the $k$-th eigenvector is computed as 
\[
(u, \phi_k)=\int_0^1u(x)\phi_k(x)=\int_0^1 U_i\varphi_i(x) F^k_j\varphi_j(x) =
U_i \left[\int_0^1 \varphi_i(x) \varphi_j(x)\right] F^k_j.
\]
The bracketed term are the entries of the mass matrix $M$. Putting elements $U_i$ into vector $U$ and similarly for $F^K_j$ to form vector $F^k$, the inner
product can be written as $(u, \phi_k) = U\cdot M\cdot F^k$.
Finally the $L^2$-norm of $u$ is 
\begin{equation}
\label{eq:l2norm}
||u||=\sqrt{(U\cdot M\cdot F^k)^2}.
\end{equation}

Kent's code would simplify the mass matrix as $M=hI$ where $h$ was the mesh
size $h=1/n$. The simplification is based on using the midpoint quadrature to
evaluate the integral.This is the reason why the $L^2$-norm has the  \pythoninline{len(ex)} term; it is the $dx$ or the length of the cell.

\begin{python}
sqrt(ex.T*l**0*ex) / len(ex)
\end{python}

Approximating the mass matrix by identity matrix leads to bigger errors if
higher order Lagrange elements are used. This is shown in Table
\ref{table:comparison}. The error is especially big if we look at $H^1_0$-norm.
Let us now see how that quantity is computed.

Multiplying (\ref{p1}) by the test function $v\in V$ leads to a variational 
formulation of the eigenvalue problem: Find $u\in V$, $\lambda\in\mathbb{C}$
such that
\begin{equation}
\label{eq:eigen}
(u_x, v_x) = \lambda(u, v),\quad\forall v\in V.
\end{equation}
Note that the finite element discretization of the variational problem would 
lead to the generalized eigenvalue problem $AU=\lambda MU$. Computation of 
the $H^1_0$-norm is based on (\ref{eq:eigen}) and the ideas explained in the
previous section. We get
\begin{equation}
\label{eq:final}
||u_x||^2=(u_x, u_x)=c_kc_i(\phi_{k,x}, \phi_{i,x})=
c_kc_k\lambda^k(\phi_k, \phi_i)=
c_kc_k\lambda^k=\lambda^k(U\cdot M\cdot F^k)^2.
\end{equation}

Formulas (\ref{eq:l2norm}) and (\ref{eq:final}) are mirrored in the code shown
at the end of the text. The code produce results in Table \ref{table:comparison}
which are in excellent agreement with norms produced by integration. The 
errors can be attributed to errors of eigenvalues/eigenvector calculations
that are due to octave routines. For convenience and easier comparison, code
using the approximate quadrature is also attached. 

\begin{table}
\begin{center}
 \caption{Comparison of errors of the $L^2$ and $H^1_0$ norms of $\sin{\pi x}$.
 The error is the difference between eigenvalue based norm and the norm computed by finite element integration. Results in the first two columns are
 obtained with code that uses midpoint rule for all integrations. The last two
 columns show results of code that uses mass matrix. 
 }
  \label{table:comparison}
 \begin{tabular}{ |c|c|c|c|c|}
   \hline
   -- & \multicolumn{2}{c|}{Inexact} & \multicolumn{2}{c|}{Exact}\\
   \hline
    p & $H^1_0$ & $L^2$ & $H^1_0$ & $L^2$\\
   \hline
		1 & 4E-3  & 1E-3 &   2E-11     &    1E-16 \\
		2 & 1E3   & 1E-1 &   4E-10     &    2E-16 \\
		3 & 6E2   & 2E-2 &   9E-10     &  9E-16 \\
   \hline
 \end{tabular}
\end{center}
\end{table}

\newpage
\pythonexternal{eig_kent.py}{\begin{center}Code using midpoint rule.\end{center}}
\newpage
\pythonexternal{eig_powers.py}{\begin{center}Code using exact integration. rule.\end{center}}
\end{document}
